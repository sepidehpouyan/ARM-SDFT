\section{Related Work}\label{sec:related3}

\paragraph{\textbf{Side-channel Attacks on TrustZone-M}}
%
Extensive research has been conducted on microarchitectural timing channels
\cite{timingattack}, notably introduced by Kocher \cite{Kocher96}, gaining
widespread attention following the disclosure of Spectre \cite{spectre} and
Meltdown \cite{meltdown}. However, exploration into side-channel attacks
within TEE context is a relatively recent endeavor.  Several authors
\cite{loadstep, truspy, Bypassed, Qualcomm, vanbulckphdthesis,
gross2019breaking, surveyonTEE} have raised concerns regarding software
side-channel vulnerabilities in higher-end TEEs like ARM TrustZone.
Additionally, research efforts on Microcontroller Units (MCUs)
\cite{Nemesis, marton, busted, returntononsecure, oflynn2019ondevice,
barenghi2021cortexm, gnad2019leakynoise} have investigated the potential
for information leakage through software-based side-channels. For instance,
Gnad et al. in \cite{gnad2019leakynoise} capitalized on the correlation
between ADC noise and MCU power consumption in Cortex-M4, utilizing
software power consumption traces to extract secret keys from an AES
implementation. Similarly, O'Flynn and Alex Dewar in
\cite{oflynn2019ondevice} exploited the ADC in a SAM L11 (Cortex-M23) MCU,
executing a remote power side-channel attack to bypass TrustZone-M
protection and retrieve a secret key. In contrast to power side-channel
attacks, Nemesis attack by Van Bulck et al. \cite{Nemesis} exploits the
CPU's interrupt mechanism to extract instruction timings from MSP430 MCUs.
In \cite{marton}, the authors leverage minor timing variations in
unprivileged DMA requests, arising from contention on the shared memory bus
within openMSP430 MCUs, to acquire a memory access trace of a victim
program. Likewise, BUSted \cite{busted} represents a type of side-channel
attack utilizing timing discrepancies on the MCU bus interconnect to bypass
the security assurances provided by memory protection primitives in ARMv8-M
MCUs with TrustZone-M.
 
\paragraph{\textbf{Identifying Microarchitectural Timing Side Channels
through Static Analysis}}
%
There exists substantial literature on timing side channel detection
employing ML models \cite{MLforSC, chiappetta2016realtime,
allaf2017comparison}, dynamic taint analysis \cite{graa2017detection},
fuzzing \cite{nilizadeh2018diffuzz}, Abstract interpretation
\cite{kopf2012automatic, doychev2015cacheaudit}, Logical reduction
\cite{chen2017precise}, type-based solutions \cite{MantelAVR, scfmsp,
barthe2014system, rodrigues2016sparse, zhang2012languagebased,
lux2011tool}, and several other methodologies \cite{timingattack,
akram2020sherlock, szefer2019survey}. Closely related to our work are the
Side Channel Finder tools~\cite{lux2011tool, MantelAVR, scfmsp}, which
involve type systems for the static detection in Java programs and AVR and
MSP430 programs. To enhance the accuracy of our analysis from previous
research and expand our capability to trace information flow across
TrustZone-provided protection domains, the approach in this paper employs a
symbolic execution-based analysis.  This allows for meticulous control over
memory operations, refining the precision of our analysis.

\paragraph{\textbf{Symbolic Execution}}
%
Some works \cite{binsec, pitchfork, sung2018canal,
chattopadhyay2018symbolic, brotzman2019casym, brennan2018symbolic,
yavuz2022encider, pasareanu2016multi, bang2016string} have, furthermore,
focused on detecting microarchitectural side-channel vulnerabilities using
symbolic execution. For instance, Bang et al. \cite{bang2016string} use
symbolic execution, string analysis, and model counting to quantify leakage
for a particular type of side channel. Pasareanu et al.
\cite{pasareanu2016multi} proposed a symbolic execution approach for
side-channels detection and quantification. They measure side-channel
leakage by creating specific public inputs that trigger maximum leakage.
This is accomplished through Max-SMT solving applied to the constraints
derived from symbolic execution. ENCIDER \cite{yavuz2022encider} employs
dynamic symbolic execution and taint analysis to uncover timing and cache
side-channel vulnerabilities within Intel SGX applications. It decomposes
side-channel requirements based on the bounded non-interference property
and implements byte-level information flow tracking through API modeling.
CoCo-Channel \cite{brennan2018symbolic} employs taint analysis to detect
secret-dependent conditional statements within Java programs. It assigns
symbolic cost expressions to various program paths and utilizes symbolic
execution to identify and report paths demonstrating secret-dependent
timing behavior.

Additionally, various other studies \cite{sung2018canal,
chattopadhyay2018symbolic, brotzman2019casym} leverage symbolic execution
to derive a symbolic cache model and verify that the cache behavior remains
independent of sensitive data. Scalability concerns often hinder symbolic
execution. Daniel et al. \cite{binsec} introduced an automatic, efficient
binary-level verification method tailored for constant-time analysis. This
method conducts both bug identification and bounded verification on
practical cryptographic implementations. Employing relational symbolic
execution with specialized optimizations in information flow and
binary-level analysis, their approach maximizes shared information between
executions following the same path. Pitchfork \cite{pitchfork} unites
symbolic execution and dynamic taint tracking to accurately propagate
secret taints across all execution paths.
% , highlighting tainted branch
%conditions or memory addresses. Notably, Pitchfork can analyze
%protocol-level code by abstracting the implementation details of primitives
%through function hooks, allowing separate analysis of these components. 

%Developing constant-time code presents complexity due to the need for
%intricate low-level operations that diverge from conventional programming
%practices. Maintaining this approach proves challenging as compiler
%optimizations often fail to preserve such implementations. Moreover , the
%vulnerability revealed in the attack on the constant-time implementation
%applied to the Curve25519 elliptic curve \cite{kaufmann2016constanttime}
%highlights the error-prone nature of writing such code.

This research explores a static method for identifying timing side channels
by integrating symbolic execution and taint analysis. To the best of our
knowledge, \tool{} is the first static analysis tool capable of
automatically detecting timing side channels, Nemesis, BUSted, and covert
storage leakage, within ARM-M binaries.
