
% \section*{Preamble}
% 
% Confidential computing, with a focus on fortifying the integrity of code
% and data actively in use, has prominently embraced the deployment of
% \acp{TEE}, exemplified by technologies like ARM TrustZone. These
% \acp{TEE} leverage hardware capabilities to establish secure enclaves for
% applications, elevating the security posture. However, despite the robust
% security framework provided by \acp{TEE}, the persistent threat of
% side-channel attacks remains a formidable challenge, operating beyond
% conventional threat model boundaries. Successful exploitation of these
% attacks can compromise the default security assurances embedded in the
% underlying hardware.
% 
% In the realm of embedded systems, ARM microcontrollers have emerged as
% pivotal components. Their ubiquity in the embedded systems market can be
% attributed to their versatility, energy efficiency, scalability, industry
% support, and integration capabilities. These factors collectively make
% ARM-based solutions a preferred choice for a wide range of applications,
% contributing to their widespread adoption across various industries. Recent
% strides in enhancing the security of ARM microcontrollers have seen the
% integration of TrustZone, a technology designed to compartmentalize and
% secure sensitive computations. Nevertheless, ARM microcontrollers equipped
% with TrustZone have become enticing targets for side-channel attackers.
% This chapter explores this duality — the amalgamation of heightened
% security through TrustZone and the persistent vulnerability to side-channel
% attacks.
% 
% Building upon the side-channel detection tool~\cite{scfmsp}, this chapter takes a significant step forward to enhance
% the precision of our previous static analysis tool. It leverages a
% groundbreaking symbolic taint-tracking approach to conduct static analysis
% on ARMv8-M binaries, proactively identifying both timing and storage
% channels. We introduces \ac{SCF}\textsuperscript{ARM}, an advanced and
% precise side-channel analysis tool designed to identify information leakage
% stemming from timing side-channels, interrupt-latency attacks (commonly
% known as Nemesis), novel DMA-based attacks (referred to as BUSted), and
% unintended information flow within TrustZone applications tailored for ARM
% microcontrollers.  The evaluation results affirm the robustness and
% scalability of \ac{SCF}\textsuperscript{ARM} in detecting vulnerabilities
% within realistic applications, thereby establishing its effectiveness as a
% proactive measure for bolstering security in ARM-based confidential
% computing environments.

\section{Introduction} \label{sect:intro3}

With the rapid proliferation of \ac{IoT} devices across domains such as
smart homes, healthcare, transportation, and industrial systems, ensuring
the security of these interconnected devices has become an utmost concern.
\ac{IoT} systems, consisting of embedded devices and networked components,
handle an abundance of sensitive data, making them prime targets for
malicious actors who seek to exploit vulnerabilities to extract information
or gain control over installations~\cite{IOTSecurity1, IOTSecurity2}. Among
the multitude of security threats, timing side-channel attacks have emerged
as a significant and pervasive challenge. These side channels leverage
timing variations in program execution to compromise the confidentiality
and of sensitive data~\cite{timingattack, Nemesis, Cache1,
brumley2011remote, Travis, busted}.  

ARM-family processors have emerged as a dominant choice for embedded
devices, the \ac{IoT}, and mobile phones, capturing a substantial market
share of over 60\%\cite{arm_qualcomm}. To enhance security, ARM has
incorporated TrustZone~\cite{TZM, DemystifyingAT}, a hardware-based
\acp{TEE}, into their processors. TrustZone ensures the isolation of
security-critical software and data from the rest of the system, enabling
secure execution of critical tasks and protection of sensitive information.
It achieves this by dividing the processor into two separate and concurrent
security realms or worlds: the 'Normal World' and the 'Secure World.' These
worlds operate independently of each other, possessing distinct memory
spaces and execution environments. Thus, developers often rely on the
presumption that secrets are protected within the secure world due to the
processor's isolation guarantees.  However, research~\cite{surveyonTEE,
DemystifyingAT, loadstep, truspy, Bypassed, Qualcomm, busted} reveals the
potential vulnerability of the TrustZone secure world to side-channel
attacks that can lead to the unintended disclosure of secrets. 

%For instance, TruSpy \cite{truspy} exploits the cache contention between
%normal world and secure world to implement a timing-based cache
%side-channel attack and then extract a full 128-bit AES encryption key
%stored in the trusted environment. The research demonstrated that while
%the contents of the processor cache are safeguarded by the hardware
%isolation, the access pattern to these cache lines remains unprotected.
%Similarly, in \cite{Qualcomm}, researchers targeted Arm TrustZone in a
%malicious OS scenario. They leveraged the OS's capabilities to invoke
%interrupts and utilized the Prime+Probe \cite{primeandprobe} technique to
%recover a 256-bit private key from Qualcomm's ECDSA algorithm. %

The TrustZone technology employed in Armv8-M processors (such as
Cortex-M23/ M33/ M35P/ M55/ M85), do not claim to protect against
side-channel attacks~\cite{armdeveloper}. Primary attack vecturse here are
secret-dependent control flow with measurable timing differences or
secret-dependent memory access patterns.  Furthermore, TrustZone cannot not
effectively prevent secret leakage that stems from program implementation
flaws, which can arise from weaknesses in protocols or algorithms, as well
as mistakes made by developers.
%
%\todo{add a few references here} 

%As an example, let's consider an One-Time Password \gls{OTP} system
%implemented within the TrustZone environment \cite{trustotp}. In a secure
%and well-implemented \gls{OTP} system, once an \gls{OTP} is utilized, it
%should immediately become invalid and should not be stored in any
%accessible location. However, if the \gls{OTP}s are stored in an insecure
%manner, such as being logged or stored in plaintext on unprotected memory
%or external I/O, an unauthorized attacker who gains access to the system
%or the logs could retrieve the previously used \gls{OTP}s. %

Early detection of side channel attacks enables proactive mitigation
measures to be implemented. Over the years, researchers and practitioners
have proposed various approaches to analyze binary code, or source code
employing techniques such as symbolic execution~\cite{binsec, pitchfork},
type systems~\cite{scfmsp, MantelAVR, Agat, barthe2014system}, and machine
learning~\cite{MLforSC}, among others~\cite{timingattack}. These approaches
aim to identify and mitigate timing side channel vulnerabilities targeting
different architectures. However, each approach carries its own limitations
and strengths, necessitating a thorough exploration of the existing body of
work in this field (refer to Section \ref{sec:related3}).

In this paper, we present a new and automated approach to detecting timing
side-channel leakage in ARM TrustZone-M programs, utilizing symbolic
execution-based analysis for the static verification. Our approach and tool
targets the ARM Cortex-M23 microcontroller, capitalizing on the
predictability of instruction execution times on hese microcontrollers.
Our objective is to ensure the absence of timing side-channel
vulnerabilities, interrupt-latency vulnerabilities (such as
Nemesis~\cite{Nemesis}), DMA-based attacks (referred to as
BUSted~\cite{busted}), and detect any undesired explicit and implicit
information flow, which is roughly equivalent to the concept of covert
storage channels in later literature~\cite{storagechannel}. This is
particularly relevant in the context of applications that are
compartmentalized into a security critical application part (such as
managing and using cryptographic credentials) and a less critical part
(such as sending and receiving network packets) to make use of the ARM
TrustZone. 

Our proposed approach is implemented in an automated tool, \tool{} to
statically detect the aforementioned vulnerabilities vulnerabilities in
ARMv8-M binaries. The primary objective of \tool{} is to track the flow of
secret information between the TrustZone's secure world and the non-secure
world, detecting and reporting any potential information leakages. \tool{}
stands for ``Side Channel Finder for ARM'' and is named after earlier tools
with similar abilities for the AVR and MSP430 platforms~\cite{scfmsp,
MantelAVR} but relies on different analysis techniques.  To the best of our
knowledge, \tool{} presents the first static analysis tool for side-channel
detection in ARMv8-M binaries, and addresses both timing and storage
channels. 

To establish the efficacy of our approach, we evaluate \tool{} on a set of
benchmarks that spanned a diverse range of synthetic benchmark programs
designed to unveil both typical and challenging structures of
secret-dependent control flow~\cite{hans},
and by running \tool{} on code from the BUSted~\cite{busted} publication.
In summary, our contributions include:

\begin{itemize}
%
  \item{We present a novel approach that relies on symbolic execution and
static analysis to conduct sound information flow analysis in compiled
ARM-V8 programs. our approach is tailored to detect side-channel
vulnerabilities in applications compartmentalization with ARM TrustZone.}
%
  \item{We have implemented our approach in a tool \tool{}, which we
evaluae and show that our approach successfully detects  timing side
channel attacks, Nemesis attacks~\cite{Nemesis}, BUSted
attacks~\cite{busted}, and undesired direct and indirect information flow
to unprotected locations. Our results demonstrate a high precision and
scalability of \tool{}, making it useful for real-world security
assessments.}
%
  \item{We make \tool{} and our benchmark datasets publicly available under
an open-souce license at \url{\toolurl}
%
%\todo{license? pipe this through \url{https://anonymous.4open.science/} for
%the submission}
}
%
\end{itemize}

