\section{Evaluation}

In this section, we present our assessment of
\tool{}, emphasizing the accuracy of our static
analysis in detecting side channels. Our evaluation of
\tool{} included a thorough examination of a
carefully selected set of benchmarks, influenced by \cite{hans}, featuring
a diverse range of programs, from those with vulnerabilities to benign
ones. These benchmarks were specifically designed to reveal typical and
intricate patterns in secret-dependent control flow, providing a
comprehensive examination of \tool{}'s effectiveness. 

The benchmarks consist of C programs compiled using 'arm-none-eabi-gcc,' a
cross-compiler toolchain tailored for generating code targeting ARM
Cortex-M and Cortex-R processors without an operating system (bare-metal).
The compiler is configured with the '-mcpu=cortex-m23' option to generate
optimized binary for ARM Cortex-M23 architecture. Here, we provide a brief
overview of selected benchmark programs. 

 \begin{figure}
  \centering
  \includegraphics[width=\columnwidth]{figures/string.jpg}
  \caption{String Comparison}
  \label{fig:string}
\end{figure}


\begin{itemize}
%
    \item \textbf{busted:} This program features a vulnerability to the 'busted' attack. The vulnerability arises from the execution of the 'str' instruction at different time offsets within the 'then' and 'else' regions. However, it is free of start-to-end timing leaks due to having a balanced branch.
%
    \item \textbf{busted-free:} This program uses \texttt{nop} to
delay the execution of the 'str' in the 'else' branch by one
clock cycle. This intentional delay ensures that memory access occurs at
the same time offset but results in unbalanced branches.
%
    \item \textbf{strcmp:} This code performs a character-by-character
comparison of two strings (cf. Fig. \ref{fig:string} (a)) and includes secret-dependent branches.
%
    \item \textbf{constant\_time\_strcmp:} Utilizing a constant-time
implementation we fix the timing side channel in our string comparison code (cf. Fig. \ref{fig:string} (b)).
%
    \item \textbf{diamond:} This program features two branching instructions dependent on secret values and one independent branch.
%
    \item \textbf{loop:} The program contains a loop that relies on a condition with public data and includes a secret-dependent branch within the loop body. 
%    
    \item \textbf{secure\_loop:} This code employs bitwise operations to
eliminate the need for the branch.
%    
    \item \textbf{multifork:} In this program, a switch statement evaluates the value of a secret variable and compares it against multiple cases. 
    
    \item \textbf{ifthenloop \& ifthenloop\_nop\_padded:} These programs
involve a branch dependent on secret data and a loop with a public
condition variable, executed if the branch is taken. In the
'padded' version, \texttt{nop}s ensure balanced branches.  
%    
    \item \textbf{ifthenlooplooptail:} This program introduces intricate
control flow with nested
secret-dependent branches and loops.
%
\end{itemize}


\begin{table*}
    \centering
		\caption{ Evaluation results for \tool{} on the benchmark
suite. For each compiled example program, we give an indication for C
program complexity (LOC = lines of code, CFG Size = number of nodes in the
program’s Control Flow Graph), execution time of
\tool{}, and list the vulnerabilities found by
\tool{}.}
  
		\label{tab:eval3}
  
		\begin{tabular}{llcccccc}
			\hline
			\textbf{Benchmark} &\textbf{LOC}
& \textbf{CFG}& \textbf{\ac{SCF}\textsuperscript{ARM}}& \textbf{Timing}& \textbf{Nemesis} &\textbf{BUSted} & \textbf{Storage}\\ 
		    {} &{} & \textbf{Size}& \textbf{exec time}& \textbf{Channel}& \textbf{Vulnerability} & \textbf{Vulnerability} & \textbf{Channel}\\ 
			\hline

busted & 14 & 20 & 3.042s & $\times$ & \checkmark & \checkmark & \checkmark \\ 

busted-free & 15 & 21 & 3.725s & \checkmark & \checkmark & $\times$ & \checkmark \\   

strcmp & 10 & 34 & 4.563s & \checkmark & \checkmark & \checkmark &      $\times$ \\ 

constant\_time\_strcmp & 13 & 39 & 2.564s & $\times$ & $\times$ &       $\times$ & $\times$ \\ 

diamond & 20 & 35 & 4.762s & \checkmark & \checkmark & \checkmark & \checkmark \\ 

loop & 16 & 49 & 3.330s & \checkmark &  \checkmark & \checkmark & \checkmark \\ 

secure\_loop & 22 & 58 & 4.902s & $\times$ & $\times$ & $\times$ & \checkmark \\ 

fork & 7 & 19 & 3.269s & \checkmark & \checkmark & \checkmark & \checkmark \\ 

fork\_nop\_padded & 12 & 24 & 3.769s & $\times$ & \checkmark & \checkmark & \checkmark \\ 

indirect & 20 & 29 & 3.460s & \checkmark & \checkmark & \checkmark & \checkmark \\ 

multifork & 15 & 34 & 4.013s & \checkmark & \checkmark & \checkmark & \checkmark \\

branchless\_multifork & 8 & 42 & 2.713s & $\times$ & $\times$ & $\times$ & \checkmark \\ 

ifcompound & 22 & 33 & 4.480s & \checkmark & \checkmark & \checkmark & \checkmark \\

call & 19 & 29 & 2.762s & $\times$ & $\times$  & $\times$ & \checkmark\\
		 		   
ifthenloop & 20 & 43 & 4.476s & \checkmark & \checkmark & \checkmark & $\times$ \\ 

ifthenloop\_nop\_padded & 25 & 50 & 4.608s & $\times$ & \checkmark & \checkmark & $\times$ \\

ifthenloopif & 28 & 74 & 6.890s & \checkmark & \checkmark & \checkmark & $\times$ \\ 

ifthenlooploop & 28 & 67 &  7.580s & \checkmark & \checkmark & \checkmark & $\times$ \\

ifthenlooplooptail & 38 & 110 & 8.136s & \checkmark & \checkmark & \checkmark & $\times$ \\

multiply & 5 & 12 & 3.789s & $\times$ & $\times$ & $\times$ & \checkmark \\
	   		   
		   \hline
		\end{tabular}
\end{table*}

In Table \ref{tab:eval3}, we present the experimental outcomes obtained by
applying \tool{} to the benchmark. Our results showcases that \tool{}
effectively analyzes intricate binaries designed for ARM Cortex-M23. The
tool identifies a spectrum of side-channel vulnerabilities
and also demonstrates its efficiency in swiftly and accurately detecting
sensitive data leakage. Branch balancing or
constant-time programming by itself cannot entirely eliminate the risk of
unintended information leakage within a program. In such instances, it
becomes imperative to take additional measures, e.g., storing
program outputs in protected memory.
% These precautions provide an extra layer of security,
% effectively mitigating the possibility of sensitive information being
% unintentionally exposed to attackers, even when employing the mentioned
% programming techniques.

\subsection{Discussion}

\paragraph{\textbf{Soundness \& Path Explosion}}
%
In the absence of
sophisticated architectural features such as paging, caches, or
out-of-order execution, the time taken for instruction
execution on th Cortex-M23 is wholly deterministic. Our approach is sound
in the sense that it follows the instruction semantics and execution times
of the processor documentation. However, our approach is incomplete due to
state and path explosion for complex programs, which would lead to early
termination of the analysis.

\paragraph{\textbf{Nemesis \& Busted Detection}}
%
As of our current knowledge, there have been no reported instances of a
Nemesis attack being executed on an ARM Cortex-M microcontroller.
Nevertheless, the ARM Cortex-M23 meets the 
the necessary prerequisites for a Nemesis attack,
completing the current instruction before
handling an interrupt~\cite{Nemesis}. This  introduces timing variations between
interrupted instructions, which could be leveraged by an attacker to
distinguish between secret-dependent branches. Furthermore, in our attacker
model, the adversary has the capability to control a timer device and
closely monitor \acp{ISR}, roughly resembling a configuration conducive to
launching a Busted attack~\cite{busted}.

% \paragraph{\textbf{Path explosion}}
% %
% Similar to other symbolic-execution tools, \tool{} encounters the widely
% recognized challenge of state explosion, particularly evident when dealing
% with larger binaries.  The exponential increase in path complexities
% renders exhaustive exploration practically infeasible, potentially leaving
% vulnerabilities undetected within unexplored paths.  Additionally, it is
% noteworthy that Angr, the foundational symbolic execution framework for
% \tool{}, introduces an inherent unsoundness by potentially concretizing
% values during symbolic execution. These challenges are orthogonal to the
% core contribution of this paper. Addressing the issue of path explosion and
% enhancing the soundness of symbolic execution tools are recognized as
% critical engineering tasks.  While beyond the scope of the current study,
% we acknowledge them as essential aspects deserving of attention in future
% work.
% 

