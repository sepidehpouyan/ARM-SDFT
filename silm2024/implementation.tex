\section{\tool{} Design \& Implementation}

We have developed the \tool{} specifically for the
static analysis of ARMv8-M binaries targeting Cortex-M23, focusing on
identifying timing side-channel vulnerabilities. Our approach, centered
around symbolic taint-tracking, enables a rapid, cost-effective, and
automated assessment of these vulnerabilities. We have implemented an open
source  version of \tool{} built upon
Angr~\cite{angr} and \tool{} \cite{scfmsp}.
Specifically, \emph{(1)} we use Angr for symbolic execution of binaries and
conducting Value Set Analysis (\ac{VSA}) to determine potential register
values or symbolic memory addresses; \emph{(2)} we expand the functionality of the
\tool{} tool to analyze TrustZone-M targeted binaries
and ensure program integrity against a novel DMA-based attack
termed BUSted \cite{busted}. 

Fig.~\ref{fig:SCFARM} showcases a high-level workflow diagram detailing the
components of the \tool{} tool. The dashed box
denotes the core elements implementing our proposed side channel evaluation
technique. \tool{} is built on approximately 1110
lines of Python code and integrates established Python libraries,
detailed below.

\begin{figure*}
  \centering
  \includegraphics[width=.9\textwidth]{figures/SCFARM.jpg}
  \caption{Overview of \tool{}}
  \label{fig:SCFARM}
\end{figure*}

We'll now provide a concise summary of each element and their practical
implementations:

\paragraph{\textbf{Parser}}
%
We employ this component to convert the input binary into
assembly language. Initially, we utilize the pyelftools Python
library~\cite{pyelftools} to locate the starting point of instructions
within the binary. Subsequently, we leverage the Capstone
disassembler~\cite{capstone} to translate the binary into ARMv8-M
instructions, generating mnemonic codes and symbolic representations of
processor instructions. This process allows us to extract
essential details such as opcodes, addresses, and instruction lengths.
Finally, we determine the required clock cycles for executing each
instruction based on the specifications outlined in the ARMv8-M
Architecture Reference Manual~\cite{armv8m_ref_manual}.

\paragraph{\textbf{CFG Generator}}
%
This component is responsible to construct an accurate program Control Flow
Graph (CFG) using the NetworkX \cite{networkX} Python library. This
component establishes a mutual connection with the subsequent component,
enabling the retrieval of the target address for ‘branch and exchange’
instruction \textit{BX <Rm>} through \ac{VSA} in Angr. The resulting graph
facilitates the identification of execution point predecessors and
successors, aiding in computing control dependence regions for branches and
inferring loops within assembly programs.

\paragraph{\textbf{Angr}} Utilizing a JSON-formatted configuration file
that lists the starting function's arguments derived from high-level code
to determine taint sources (i.e., highly sensitive inputs). This component
employs Angr's annotation system to mark registers and memory cells
containing confidential data with a taint label. It associates arguments
with their corresponding registers following the ARMv8-M calling
conventions~\cite{armv8m_ref_manual}. These taint tags propagate during
symbolic execution. Our taint analyzer tracks both
explicit and implicit flows. Upon completion of symbolic execution, the
component presents a list of tainted registers and memory cells for
subsequent analysis. 

\paragraph{\textbf{SC Checker}} This module performs static analysis to
identify potential side channel leaks within ARMv8-M assembly programs. Our
\tool{} tool currently detects vulnerabilities
falling into four distinct classes:

\begin{itemize}
%
\item Timing Channel: Detected when a program exhibits different execution
times depending on the secret input. \tool{} detects
these leaks by measuring the overall running times of if/else regions
in secret-dependent branches.
%
\item BUSted Vulnerability: Recognized when a program includes store/load
instructions with different time offsets inside the if/else regions of a
secret-dependent branch.
%
\item Nemesis Vulnerability: Identified when a program displays varying
execution times for at least one instruction within the if/else regions of
a secret-dependent branch.
%
\item Storage Channel: Flagged if there is an undesired information flow
from secret inputs to observable output, such as return registers or
non-secure memory.
%
\end{itemize}

