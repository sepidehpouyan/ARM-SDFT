\section{Conclusions and Future Directions}

We presented \tool{} as a new tool for the automated detection of
microarchitectural side channels, specifically targeting timing side
channels, Nemesis, BUSted attacks, and information leakage, in TrustZone-M
applications. Our work relies on the predictability of execution times on
Cortex-M processors, which allowed us to develop a symbolic taint-tracking
approach for timing-sensitive information flow analysis with a high degree
of precision. We applied \tool{} to a range of synthetic and real-world
benchmarks. The outcomes of these experiments illustrate the tool's
capability to identify a spectrum of side-channel vulnerabilities, enabling
developers to assess and mitigate these vulnerabilities. To the best of our
knowledge \tool{} is the first side-channel detection tool for compiled
ARMv8-M programs and we make our tool and benchmarks publicly available.

Several avenues for future research emerge from this work. Firstly, we aim
to enhance the comprehensiveness of our framework by extending its coverage
to encompass the entire ARMv8-M instruction set. We intend to conduct a
comprehensive evaluation of \tool{} by applying it to off-the-shelf
TrustZone-M programs and scrutinizing its performance in real-world
implementations of cryptographic libraries, such as wolfSSL~\cite{wolfssl}.
Moreover, in pursuit of efficiency improvements, the incorporation of
state-merging~\cite{kuznetsov2012efficient} or path prioritization
strategies~\cite{baldoni2018survey, li2013steering} would be useful.

